%\begin{landscape}
% Table generated by Excel2LaTeX from sheet 'Sheet 1'
%\begin{table}[htbp]
% \centering
 % \caption{Known and used schemes of feature weighting.}
    \begin{longtable}{p{.15\textwidth}p{.85\textwidth}}
    \caption{Known and used schemes of feature weighting.} \\
    \hline    
    \specialcell{\textbf{Aspect of work}} & \multicolumn{1}{c}{\textbf{Reference/Description}} \\
	\hline
	
	& \multicolumn{1}{c}{\textbf{~\citet{Tang2020}}} \\
    \specialcell{Details} &
    A new term weighting schemat called Frequency-inverse Exponential Frequency (TF-IEF) with the new global weighting factor IEF to characterize the global weighting factor is introduced.
    \\ 
    \specialcell{Findings} & 
    TF-IEF outperforms the term weighting schemes, such as TF-CHI2 and TF-IG.
    \\  
    \specialcell{Challenges} & 
    There is no highlighted challenges or open problems.
    \\
    
	& \multicolumn{1}{c}{\textbf{~\citet{Chen2016}}} \\
    \specialcell{Details} &
    A new Supervised Term Weighting (STW) schema called Term Frequency \& Inverse Gravity Moment (TF-IGM) is introduced.   
    \\  
    \specialcell{Findings} & 
    TF-IGM outperforms the TF-IDF and the state-of-the-art STW schemes.
    \\  
    \specialcell{Challenges} & 
    To conduct comparative studies on the IGM model as a new measure of sample distribution. To applying the IGM model to feature dimension reduction and sentiment analysis. 
	\\
	
	& \multicolumn{1}{c}{\textbf{~\citet{Luo2011}}} \\
    \specialcell{Details} & 
    The authors propose a novel term weighting scheme by exploiting the semantics of categories and indexing terms. 
    \\  
    \specialcell{Findings} & 
    The approach outperforms TF-IDF when the amount of training data is small, or the content of documents is focused on well-defined categories.    
	\\  
	\specialcell{Challenges} & 
	Other ontologies with wider coverage for expressing the senses of words and category labels should be employed. Different ways of representing the semantics of categories and other similarity measures should be explore.   
	\\
	
    \hline
     \label{tab:fw}
    \end{longtable}%
  %\label{tab:addlabel}%
%\end{table}%
%\end{landscape}