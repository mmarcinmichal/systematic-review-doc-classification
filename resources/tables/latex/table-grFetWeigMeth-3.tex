%\begin{landscape}
% Table generated by Excel2LaTeX from sheet 'Sheet 1'
%\begin{table}[htbp]
% \centering
 % \caption{Known and used schemes of feature weighting.}
    \begin{longtable}{p{.15\textwidth}p{.85\textwidth}}
    \caption{Known and used schemes of feature weighting.} \\
    \hline    
    \specialcell{\textbf{Aspect of work}} & \multicolumn{1}{c}{\textbf{Reference/Description}} \\
	\hline
	
	& \multicolumn{1}{c}{\textbf{~\citet{Tang2020}}} \\
    \specialcell{Details} &
    A new term weighting scheme called Frequency-inverse Exponential Frequency (TF-IEF), with a new global weighting factor, IEF, to characterize a global weighting factor is introduced.
    \\ 
    \specialcell{Findings} & 
    TF-IEF outperforms other term weighting schemes, such as TF-CHI2 and TF-IG.
    \\  
    \specialcell{Challenges} & 
    The authors failed to highlight any challenges or open problems.
    \\
    
	& \multicolumn{1}{c}{\textbf{~\citet{Chen2016}}} \\
    \specialcell{Details} &
    A new Supervised Term Weighting (STW) scheme called Term Frequency \& Inverse Gravity Moment (TF-IGM) is introduced.   
    \\  
    \specialcell{Findings} & 
    TF-IGM outperforms TF-IDF and the state-of-the-art STW schemes.
    \\  
    \specialcell{Challenges} & 
    Comparative studies on the IGM model as a new measure of sample distribution should be conducted. The model should be applied to feature dimension reduction and sentiment analysis. 
	\\
	
	& \multicolumn{1}{c}{\textbf{~\citet{Luo2011}}} \\
    \specialcell{Details} & 
    The authors propose a novel term weighting scheme by exploiting the semantics of categories and indexing terms. 
    \\  
    \specialcell{Findings} & 
    The approach outperforms TF-IDF with small amounts of training data, or when the content of the documents is focused on well-defined categories.    
	\\  
	\specialcell{Challenges} & 
	Other ontologies, with wider coverage for expressing the sense of words and category labels, should be employed. Different ways of representing the semantics of categories and other similarity measures should be explored.   
	\\
	
    \hline
    \label{tab:fw}
    \end{longtable}%
  %\label{tab:addlabel}%
%\end{table}%
%\end{landscape}
