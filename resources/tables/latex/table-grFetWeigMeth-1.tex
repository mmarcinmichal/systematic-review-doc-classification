%\begin{landscape}
% Table generated by Excel2LaTeX from sheet 'Sheet 1'
%\begin{table}[htbp]
% \centering
 % \caption{Known and used schemes of feature weighting.}
    \begin{longtable}{p{.15\textwidth}p{.85\textwidth}}
    \caption{Known and used schemes of feature weighting.} \\
    \hline    
    \specialcell{\textbf{Aspect of work}} & \multicolumn{1}{c}{\textbf{Reference/Description}} \\
	\hline
	
	& \multicolumn{1}{c}{\textbf{~\citet{Tang2020}}} \\
    \specialcell{Details} &
    The authors describe a new term weighting schemat called frequency-inverse exponential frequency (TF-IEF). The proposed method replaces inverse document frequency (IDF) in the frequency-inverse document frequency (TF-IDF) weighting schema with the new global weighting factor IEF to characterize the global weighting factor.
    \\ 
    \specialcell{Findings} & 
    In work, the authors show that TF-IEF outperforms the state-of-the-art term weighting schemes, such as TF-CHI2 and TF-IG.
    \\  
    \specialcell{Challenges} & 
    There is no highlighted challenges or open problems.
    \\
    
	& \multicolumn{1}{c}{\textbf{~\citet{Chen2016}}} \\
    \specialcell{Details} &
    The authors describe a new term weighting schema called Term frequency \& Inverse gravity moment (TF-IGM). The proposed method is a type of Supervised Term Weighting (STW) approach where during term weighting is incorporate information about the class.  
    \\  
    \specialcell{Findings} & 
    TF-IGM outperforms the famous TF-IDF and the state-of-the-art supervised term weighting schemes. Besides, some new findings different from previous studies are obtained and analyzed in-depth in the paper.
    \\  
    \specialcell{Challenges} & 
    The authors in the future want to conduct comparative studies on the IGM model as a new measure of sample distribution non-uniformity and the traditional statistical models such as variance and entropy. Also, the authors highlight the possibility of applying the IGM model to feature dimension reduction and sentiment analysis. 
	\\
	
	& \multicolumn{1}{c}{\textbf{~\citet{Luo2011}}} \\
    \specialcell{Details} & 
    The authors propose a novel term weighting scheme by exploiting the semantics of categories and indexing terms. The proposed term weighting approach includes by the following steps: (1) Determining the semantics of categories, (2) Estimating the semantic similarity of each term with the category, and (3) Combining the semantic similarity of each term with the category and its term frequency in a document to obtain the feature vector of each document. 
    \\  
    \specialcell{Findings} & 
    The proposed approach outperforms TF-IDF when the amount of training data is small, or the content of documents is focused on well-defined categories.    
	\\  
	\specialcell{Challenges} & 
	The authors in the future want to employ other ontologies with wider coverage (such as Wikipedia) for expressing the senses of words and category labels. Also, they plan to explore different ways of representing the semantics of categories and other similarity measures.   
	\\
	
    \hline
     \label{tab:fw}
    \end{longtable}%
  %\label{tab:addlabel}%
%\end{table}%
%\end{landscape}