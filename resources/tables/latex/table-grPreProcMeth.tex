\begin{landscape}
% Table generated by Excel2LaTeX from sheet 'Sheet 1'
%\begin{table}[htbp]
% \centering
 % \caption{Known and used pre-processing method.}
    \begin{longtable}{lp{.3\textwidth}p{.8\textwidth}}
    \caption{Known and used pre-processing method.} \\
    \hline    
    Reference & \multicolumn{1}{c}{Aspect of work} & \multicolumn{1}{c}{Description} \\
	\hline
	
    \multirow{3}[10]{*}{~\citep{Sajgalik2019}} & 
    \specialcell{Technical and algorithmic \\ aspect of the work} & 
    The authors propose a method thanks to we can extract valuable keywords from a document. An embedded document representation is created based on these keywords, and it is utilised in the classification process. First, in the document are found and extracted phrases. Second, for each phrase is computed a vector representation. Third, the authors find the most similar words (words are also embedded, i.e. they are represented as vectors) to the given phrase. These words are the final keywords. Finally,  each document is described by a set of keywords that are summed up and normalised. So, in this way is achieved the final text representation, which is utilised in the classification process. 
    \\ & 
    \specialcell{Findings/recommendations \\ of the research} & 
    The authors tried k-nearest neighbours classifier (k-nn) and tried using cosine distance as well as Euclidean distance as the distance metric. However, neither of them performed well. The authors found that from results obtained by Linear Discriminant Analysis (LDA) classifier approximate the results of Support Vector Machine (SVM) classifier quite well. The authors suggest that it is best to use SVM classifier with the 1-against-all strategy for categorising documents represented as keywords extracted by our method. The method can be used to extract a small number of keywords to obtain a human-readable form of metadata that summarises the whole document. Also, based on the extracted keywords, we can categorise a document with high precision.
	\\ & 
	\specialcell{Highlighted challenges \\ or open problems} & 
	The main limitation of the proposed method is that it requires the existence of categories which can be used to categorise documents. The authors would like to experiment with utilising the discriminative metrics earlier in the process to obtain more category-dependent similar words for a given candidate phrase. 
	\\
        
    \hline
    \label{tab:ppm}
    \end{longtable}%
  %\label{tab:addlabel}%
%\end{table}%
\end{landscape}