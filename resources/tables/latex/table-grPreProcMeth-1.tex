%\begin{landscape}
% Table generated by Excel2LaTeX from sheet 'Sheet 1'
\begin{table}[htbp]
 \centering
 \caption{Known and used pre-processing method.}
    \begin{tabular}{p{.15\textwidth}p{.85\textwidth}}
%    \caption{Known and used pre-processing method.} \\
    \hline    
    \specialcell{\textbf{Aspect of work}} & \multicolumn{1}{c}{\textbf{Reference/Description}} \\
	\hline
	
    & \multicolumn{1}{c}{\textbf{~\citet{Sajgalik2019}}} \\ 	 
    \specialcell{Details} & 
    The authors propose a method thanks to we can extract valuable keywords from a document. An embedded document representation is created based on these keywords, and it is utilised in the classification process. 
    \\  
    \specialcell{Findings} & 
    The authors suggest that it is best to use Support Vector Machine (SVM) classifier with the 1-against-all strategy for categorising documents represented as keywords extracted by their method. The method can be used to extract a small number of keywords to obtain a human-readable form of metadata that summarises the whole document.
	\\  
	\specialcell{Challenges} & 
	The main limitation of the proposed method is that it requires the existence of categories which can be used to categorise documents. The authors would like to experiment with utilising the discriminative metrics earlier in the process to obtain more category-dependent similar words for a given candidate phrase. 
	\\
        
    \hline
    \label{tab:ppm}
    \end{tabular}%
  %\label{tab:addlabel}%
\end{table}%
%\end{landscape}